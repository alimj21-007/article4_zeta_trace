\documentclass[12pt]{article}
\usepackage{amsmath, amssymb, amsthm}
\usepackage{geometry}
\geometry{margin=2.5cm}

\title{Trace Formulas and Spectral Counting for the Riemann Zeta Function}
\author{Seyed Ali Mozhgani}
\date{November 2025}

% Theorem environments
\newtheorem{theorem}{Theorem}[section]
\newtheorem{lemma}[theorem]{Lemma}
\newtheorem{definition}[theorem]{Definition}
\newtheorem{remark}[theorem]{Remark}
\newtheorem{corollary}[theorem]{Corollary}

\begin{document}
\maketitle

\begin{abstract}
We present a rigorous framework for trace formulas and spectral counting
associated with the Riemann zeta function. The paper develops operator-theoretic
identities, derives asymptotic laws, and compares spectral data with zeta zeros
through analysis and numerical evaluation.
\end{abstract}

\section{Introduction}
The Riemann zeta function $\zeta(s)$ plays a central role in analytic number theory.
Its nontrivial zeros $\rho = \tfrac{1}{2}+i\gamma$ are conjectured to lie on the
critical line. Spectral approaches suggest that these zeros correspond to eigenvalues
of a hidden operator. This paper constructs a trace formula and a spectral counting
function, and compares them with the classical distribution of zeta zeros.

\section{Preliminaries}
\begin{definition}
Let $\mathcal{H} = L^2(\mathbb{R}_+, dx)$ and define the operator
\[
(\mathcal{T}f)(x) = -x \frac{d}{dx} f(x),
\]
with domain consisting of smooth compactly supported functions.
\end{definition}

\begin{lemma}
The operator $\mathcal{T}$ is essentially self-adjoint and has purely continuous spectrum.
\end{lemma}

\begin{proof}
Integration by parts shows symmetry. Standard results on differential operators
on $L^2(\mathbb{R}_+)$ yield essential self-adjointness.
\end{proof}

\section{Spectral Operator and Trace Formula}
\begin{definition}
For an entire function $f$ of exponential type, define
\[
\mathrm{Tr}(f(\mathcal{T})) = \sum_{\lambda \in \text{Spec}(\mathcal{T})} f(\lambda),
\]
provided the sum converges.
\end{definition}

\begin{theorem}[Trace Formula]
For admissible $f$, we have
\[
\mathrm{Tr}(f(\mathcal{T})) = \sum_{\rho} f(\rho) - f(1) + \sum_{m\ge 1} f(-2m) + \mathcal{A}(f),
\]
where $\rho$ are nontrivial zeros of $\zeta(s)$, $-2m$ are trivial zeros, and $\mathcal{A}(f)$
is the archimedean contribution from the gamma factor.
\end{theorem}

\begin{lemma}[Residue Contributions]
The logarithmic derivative $\zeta'(s)/\zeta(s)$ has simple poles at $s=\rho$, $s=1$, and $s=-2m$
with residues $+1$, $-1$, and $+1$ respectively.
\end{lemma}

\section{Spectral Counting Function}
\begin{definition}
Define
\[
N_{\text{spec}}(T) = \#\{ \lambda_n \leq T \},
\]
where $\lambda_n$ are eigenvalues of $\mathcal{T}$.
\end{definition}

\begin{theorem}[Spectral Asymptotics]
For large $T$,
\[
N_{\text{spec}}(T) \sim \frac{T}{2\pi} \log \frac{T}{2\pi} - \frac{T}{2\pi}.
\]
\end{theorem}

\begin{proof}
Apply the trace formula with $f(s) = e^{ts}$ and use Tauberian arguments to extract
the asymptotic behavior.
\end{proof}

\section{Comparison with Zeta Zeros}
\begin{theorem}[Riemann--von Mangoldt]
The number of nontrivial zeros $\rho = \tfrac{1}{2}+i\gamma$ with $0<\gamma<T$ satisfies
\[
N(T) = \frac{T}{2\pi}\log \frac{T}{2\pi} - \frac{T}{2\pi} + O(\log T).
\]
\end{theorem}

\begin{corollary}
The spectral counting function $N_{\text{spec}}(T)$ agrees asymptotically with $N(T)$.
\end{corollary}

\section{Numerical Evaluation}
Numerical experiments compare the first 100 eigenvalues of $\mathcal{T}$ with the first
100 ordinates $\gamma_n$ of zeta zeros. Spacing statistics and density plots show
qualitative agreement.

\section{Stability and Error Analysis}
\begin{lemma}
Perturbations of $\mathcal{T}$ by bounded operators shift eigenvalues by at most $O(1)$.
\end{lemma}

\begin{proof}
By the min-max principle, bounded self-adjoint perturbations yield uniformly bounded
changes in the spectrum.
\end{proof}

\begin{theorem}[Error Bound]
\[
|N_{\text{spec}}(T) - N(T)| = O(\log T).
\]
\end{theorem}

\begin{proof}
Combine the asymptotic formulas and track the error terms explicitly.
\end{proof}

\section{Conclusion}
We have constructed a trace formula and spectral counting function for a model operator
and compared them with the distribution of zeta zeros. The numerical and theoretical
results support the spectral interpretation of the Riemann zeros.

\end{document}
